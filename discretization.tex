\documentclass{article}
\usepackage{amsmath,amsthm}


\begin{document}
\section{Introduction}

Consider a discrete time continuous valued dynamical system $x_{t+1} =
f(x_t,u_t)$, where $x_t$ are the state vectors and $u_t$ are the
control vectors generated by a controll. The controller's goal is to
generate the sequence of controls to minimize the cost $\sum_{t=1}^T
c(x_t)$.

To identify an optimal control strategy, we discretize the state space
and the control space and use well-understood techniques for
discrete-time discrete-valued dynamical systems.

A finer discretization mimics the continuous system more accurately
and can produce better control signals.  But control algorithms scale
at least linearly in the state space, which itself grows with the
product of the number of increments along each dimension of $x$.  I
Instead of refining the discretization, I propose coarsening the time
steps.


\section{Illustration}

To investigate the issues, consider a second order discrete time,
continuous-valued system
\begin{eqnarray*}
  x_{t+\delta_t} &= x_t + s_t \;\delta_t \\
  s_{t+\delta_t} &= s_t + a_t \;\delta_t
\end{eqnarray*}
where the exogenous acceleration $a_t$ is the control signal.

Let $\delta_x$, $\delta_s$, and $\delta_a$ be increments between
discrete values of these variables and impose bounds $|x_t|<M_x$,
$|s_t|<M_s$, and $|a_t|<M_a$. Denote by $\hat{x}$, $\hat{s}$ and
$\hat{a}$ the corresponding discretized versions of these variables
($\hat{x}$ is an integer multiple of $\delta_x$, and so on).

A very coarse discretization has several subtle problems. If $\delta_s
> M_a\delta_t$ or $\delta_x > M_s\delta_t$, no control sequence $a_t$
can affect the state and the system is stuck. Thus we have an upper
bound on $\delta_s$ and $\delta_x$.  We can also bound the number of
discrete states required to avoid this issue. Per the bound, pick a
positive constant $c<1$ and set $\delta_s = cM_a\delta_t$ and
$\delta_x = c M_s\delta_t$. The total number of states required to
represent $\hat{x}$ and $\hat{s}$ is
$\frac{M_x}{\delta_x}\frac{M_s}{\delta_s} = c^{-2}
\frac{M_x}{M_a}$. The ratio $\frac{M_x}{M_a}$ is a dynamic range
between the control signal and the attainable range of positions.

Finer discretization encounters another problem, even when
$\delta_s\to 0$ so that $\hat{s}(t)=s(t)$. Indeed, suppose the speed
is constant $s(t) = (i+\tfrac{1}{2}-\epsilon) \delta_x$ for some
integer $i$ and $x(0)=0$.  Then $x(t) = t s(t) \delta_t$ whereas
$\hat{x}(t) = i \delta_x$. The difference between $x(t)$ and
$\hat{x}(t)$ is at least $t (\tfrac{1}{2}-\epsilon)\delta_x$, which
grows linearly over time.

We're encountering roundoff errors caused by discretizing {\it values}
coarsely. Roundoff problems that accrue over time are inevitable. To
mitigate these issues, rather than discretizing vaues more finely, we
discretizing {\it time} more coarsely.

\section{Coarser Time Steps to Reduce Roundoff Errors}

Assume $a_t$ is constant on the time interval $[t\ldots,t+n\delta_t]$.
Integrating the dynamics numerically gives
\begin{eqnarray*}
  x_{t+n \delta_t} &= x_t + s_t \; \tfrac{n(n+1)}{2} \delta_t \\
  s _{t+n \delta_t } &= s _t + a _t \;n \delta_t
\end{eqnarray*}
{\bf xxx double check the quadratic term}

To avoid the first problem, it suffices to set $\delta_s = c M_a n \delta_t$ and $$
\end{document}
